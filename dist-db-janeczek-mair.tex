\documentclass[11pt,a4paper]{article}
\usepackage[utf8]{inputenc}
\usepackage{amsmath}
\usepackage{amsfonts}
\usepackage{amssymb}
\usepackage{graphicx}
\usepackage{booktabs}
\usepackage[left=2cm,right=2cm,top=2cm,bottom=2cm]{geometry}
\usepackage{authblk}
\title{\bf DEZSYS02 - DISTRIBUTED DATABASES}
\author{Janeczek Christian, Mair Wolfgang} 
\affil{IT Department TGM, Vienna}
\date{\today{}, Vienna}

\begin{document}

\maketitle
\newpage
\tableofcontents
\newpage

\section{Task Description}
Work in group of 2 and implement a SSI/SPI Master-Slave communication between two TI Tiva C Launchpads. It should be possible to toggle the green and red led on the master with the buttons of the slave device. Don't forget to de-bounce the buttons and ensure that only if the button is pressed for exactly 2 seconds the led should be toggled.\cite{SPI_SSI}


\section{Technology}
\subsection{SPI}
SPI, which means Serial Peripheral Interface, is an Interface bus which is able to send data between micro-controllers and small peripherals. To do so it uses a separate clock line and data-lines. It also has a select line in order to be able to communicate with more than one device (slave-select).\cite{SPI_expl}


\subsection{SSI}
Beispiel Text
\newpage

\section{Design Consideration}
\subsection{Nedded functions}
We use interrupts to reduce the amount of work for the master. We also need a timer to be able to execute the task at "exact" 2 seconds. After talking with the Prof. Borko about this feature he revoked the feature and changed it to about 2 seconds so a normal timer would work. We are going to use SSI to send the needed information and data.
\subsection{Solution}
In our design the master device will wait for a button to be pressed. The button-press triggers an interrupt which starts a timer. If the button-state (pressed) isn't going to be changed after about 2 seconds the timer will trigger another interrupt which sends data to the slave-device. After the slave receives the data he changes the state of his current led (on-off / off-on).
\subsection{Data definition}
The cable we are going to use only has a bandwidth of 4 to 16 Bits. This is why we need to think about what information we are going to send to the slave.

We do need a Device ID (Which device is in which row) - max. (8) devices - 3 Bits needed
We also want to transmit the port and pin info - 3 Bits for ports (6) – 3 Bits for pins (8)
The state definition of the lamp is also needed in this project - 1 Bit (2) to toggle the lamp

We also would be able to send the timer data, but we decided that the timer will run on the master side.

So we need 10 bits to send the slave the information he needs.

\newpage

\section{Apportionment of work with effort estimation}

\subsection{Task execution time}
\begin{table}[h]
\begin{tabular}{@{}|c|c|c|c|@{}}
\toprule
\textbf{Name}         &\textbf{Task}         	 & \textbf{Date} & \textbf{Duration} \\ \midrule
Wolfgang Mair         &Preparing the environment & 13-10-2014    & 1h                \\ \midrule
Christian Janeczek    &Preparing the environment & 13-10-2014    & 1h                \\ \midrule
Wolfgang Mair         &Documentation             & 14510-2014    & 1h             \\ \bottomrule
\end{tabular}
\end{table}

\bf Time worked in total = 3 h

\subsection{Estimated Work}

\begin{table}[h]
\begin{tabular}{@{}|c|c|c|@{}}
\toprule
\textbf{Task}         & \textbf{Estimated Time} & \textbf{Actual Time} \\ \midrule
Preparing environment & 1/2h                    & ?h                   \\ \midrule
Information gathering & 4h                      & ?h                   \\ \midrule
Coding Master         & 2h                      & ?h                   \\ \midrule
Coding Slave          & 2h                      & ?h                   \\ \bottomrule
\end{tabular}
\end{table}

\bf Time expected in total = 8.5 h

\newpage

\section{Task Execution}
\subsection{Creating an ER-Diagram}
<insert ERD here>

\subsection{Creating a Relational Model}
MenuHistory(\underline{date}, Menu.name1, Menu.name2, Menu.name3)Menu.name Pro Tag 3 versch. Menüs \\
Menu(\underline{name}, preis) \\
MenuSpeisen(\underline{Speise.id}, \underline{Menu.name}) \\
Speise(\underline{id}, typ, name) \\
SpeiseZutaten(\underline{Speise.id}, \underline{Zutaten.id}, menge) \\
Zutaten(\underline{id}, name, preis, einheit) \\
ZutatenLieferant(\underline{Lieferant.uid}, \underline{Zutaten.id}) \\
Lieferant(\underline{uid}, adresse, knr) \\
Lager(\underline{id}\part{title}, \underline{Zutaten.id}, bestand) \\
Bestellposten(\underline{Zutaten.id}, \underline{Bestellung.nr}, \underline{Lieferant.uid}, preis, menge) \\
Bestellung(\underline{bnr}, bestelldatum, lieferdatum) \\
Rechnung(\underline{rnr}, \underline{Bestellung.bnr}, bankv, rSumme) \\

\subsection{Coding Master}
Code snippets and special parts of the code
\subsection{Coding Slave}
Code snippets and special parts of the code
\newpage

\section{Test Report}
\newpage

\bibliographystyle{IEEEtran}
\bibliography{References}

\end{document}
