\documentclass[11pt,a4paper]{article}
\usepackage[utf8]{inputenc}
\usepackage{amsmath}
\usepackage{amsfonts}
\usepackage{amssymb}
\usepackage{graphicx}
\usepackage{booktabs}
\usepackage[left=2cm,right=2cm,top=2cm,bottom=2cm]{geometry}
\usepackage{authblk}
\title{\bf DEZSYS02 - DISTRIBUTED DATABASES}
\author{Janeczek Christian, Mair Wolfgang} 
\affil{IT Department TGM, Vienna}
\date{\today{}, Vienna}

\begin{document}

\maketitle
\newpage
\tableofcontents
\newpage

\section{Task Description}
404 error

\section{Technology}
\subsection{Distributed Databases}
404 error

\newpage

\section{Design Consideration}
\subsection{Creating an Entity-Relationship-Diagram}
The first task would be to create an Entity-Relationship-Diagram to begin with. From this point on out we are able to discuss our current design concept. The next step would be to implement the whole idea into the already downloaded OracleXE Virtual Machine. But before that, the second task has to be fulfilled: you know it: Creating a Relational Model.
\subsection{Creating a Relational Model}
The creation of the Relational-Model had to be done simultaneously to the creation of the ER-Diagram to save as much time humanly possible. The underline command will be very useful here to differ between Primary Keys and Foreign Keys. Creating the Relational Model will be an easy task, because we have direct intercommunication with each person working on this very application.
\subsection{Implementing the whole concept in OracleXE}
\subsubsection{Configuring OracleXE}
First of all, we had to get the Virtual Machine for OracleXE. This very Virtual Machine had already configured a functional OracleXE environment to work with. In this particular case we chose the password 'oracle'.
\subsubsection{Creating a new Database User}
The next step would be to create a new database user, baptised under the name 'janemair'. Put in the following command: \\ \\ \textbf{create user janemair identified by schueler;}
\subsubsection{Granting Privileges to janemair}
As we know it from other Database-Management-System like postgreSQL and mySQL, we have to grant the newly created user some of the most important privileges. This is achieved by the already known command: \\ \\ \textbf{grant all privileges to janemair;} \\ \\
\subsection{Getting started with SQL Developer}
\subsubsection{Installing SQL Developer}
To install and start SQL Developer use the following link referenced as: [1...http://www.oracle.com/technetwork/developer-tools/sql-developer/]

\newpage

\section{Apportionment of work with effort estimation}

\subsection{Task execution time}
\begin{table}[h]
\begin{tabular}{@{}|c|c|c|c|@{}}
\toprule
\textbf{Name}         &\textbf{Task}         	 & \textbf{Date} & \textbf{Duration} \\ \midrule
Wolfgang Mair         &Preparing the environment & 13-10-2014    & 1h                \\ \midrule
Christian Janeczek    &Preparing the environment & 13-10-2014    & 1h                \\ \midrule
Wolfgang Mair         &Documentation             & 14510-2014    & 1h             \\ \bottomrule
\end{tabular}
\end{table}

\bf Time worked in total = 3 h

\subsection{Estimated Work}

\begin{table}[h]
\begin{tabular}{@{}|c|c|c|@{}}
\toprule
\textbf{Task}         & \textbf{Estimated Time} & \textbf{Actual Time} \\ \midrule
Preparing environment & 1/2h                    & ?h                   \\ \midrule
Information gathering & 4h                      & ?h                   \\ \midrule
Coding Master         & 2h                      & ?h                   \\ \midrule
Coding Slave          & 2h                      & ?h                   \\ \bottomrule
\end{tabular}
\end{table}

\bf Time expected in total = 8.5 h

\newpage

\section{Task Execution}
\subsection{Creating an ER-Diagram}
<insert ERD here>

\subsection{Creating a Relational Model}
MenuHistory(\underline{date}, Menu.name1, Menu.name2, Menu.name3)Menu.name Pro Tag 3 versch. Menüs \\
Menu(\underline{name}, preis) \\
MenuSpeisen(\underline{Speise.id}, \underline{Menu.name}) \\
Speise(\underline{id}, typ, name) \\
SpeiseZutaten(\underline{Speise.id}, \underline{Zutaten.id}, menge) \\
Zutaten(\underline{id}, name, preis, einheit) \\
ZutatenLieferant(\underline{Lieferant.uid}, \underline{Zutaten.id}) \\
Lieferant(\underline{uid}, adresse, knr) \\
Lager(\underline{id}\part{title}, \underline{Zutaten.id}, bestand) \\
Bestellposten(\underline{Zutaten.id}, \underline{Bestellung.nr}, \underline{Lieferant.uid}, preis, menge) \\
Bestellung(\underline{bnr}, bestelldatum, lieferdatum) \\
Rechnung(\underline{rnr}, \underline{Bestellung.bnr}, bankv, rSumme) \\

\subsection{Coding Master}
Code snippets and special parts of the code
\subsection{Coding Slave}
Code snippets and special parts of the code
\newpage

\section{Test Report}
\newpage

\bibliographystyle{IEEEtran}
\bibliography{References}

\end{document}
